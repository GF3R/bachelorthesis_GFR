\chapter*{Version 2}
\label{chap:Technical CHallenges}
\renewcommand{\thesection}{\arabic{section}}
\setcounter{section}{0}

After my first attempts and some discussion with my supervisor I realised that the system needs to be distributed. This means that the sensors send their data wirelessly. This enables me to add additional sensors without having issues with the Wire library or physical wires. However this increases the cost and complexity of each sensor slightly, since each sensor needs some sort of "brain", which handles the connection and communication. Which means each sensor needs to be equipped with a microcontroller, for example an ESP32. These are needed to establish and manage the wireless connection.
However it reduces the complexity of the construction and the entire solution drastically. The logic will be distributed to many smaller services rather than an entire blob of logic. 

\section{Attempt A - Local Client Server Network}

The communication was achieved via WIFI since I know the protocol and can get it running swiftly. Also the esp32 I have been using has WIFI natively enabled. Connecting the devices using bluetooth or bluetooth low energy (BLE) might help reduce electricity consumption. This will need testing in the future and will not be further investigated in this thesis.

An issue with wireless communication is the synchronisation of the data flow. To simplify the process and due to lack of know how and time to achieve a correct synchronisation I will ignore it, while knowing the data might be time shifted. 
This might be an issue when visualising and analysing the data, but will be tackled then.

Furthermore for the communication via WIFI a simple "Client Server" protocol was used which has already been implement in ESP32 Micro-controllers. This protocol, as far as I understood, communicated on the transport layer using TCP.\cite{arduinoi4:online} \cite{ESP32Ser71:online}

Server Client Request: 
\begin{lstlisting}
String clientRequest(String input)
{
  Serial.println(input);
  String response = "\0";
  for (int i = 0; i < NUM_CLIENTS; i++)
  {
    WiFiClient client = server.available();
    client.setTimeout(50);
    if (client) {
      if (client.connected()) {
        client.println(input);
        data = client.readStringUntil('\r');  // received the server's answer
        Serial.println(data);
        if (data != "\0")
        {
          int Index = data.indexOf(':');
        
          CLIENT = data.substring(0, Index);
          ACTION = data.substring(Index + 1);
          Serial.println(data);
   
          if (CLIENT == "ACK")
          {
            response = ACTION;
          }

          //client.flush();
          //data = "\0";
        }
      }else{
        Serial.println("client not connected");
      }
    }
    else{
        Serial.println("client null or false");
      }
  }
\end{lstlisting}
\cite{ESP32Ser71:online}

Client Loop:

\begin{lstlisting}
void loop () {
  if (!client.connect(server, 80)) {
    while (WiFi.status() != WL_CONNECTED) {
      Serial.print(".");
      delay(500);
    }
    Serial.print("+");
    delay(100);
    return;
  }
  data = client.readStringUntil('\r');  // received the server's answer
  Serial.println(data);
  if (data != "\0")
  {
    int Index = data.indexOf(':');

    CLIENT = data.substring(0, Index);
    ACTION = data.substring(Index + 1);

    if (CLIENT == CLIENT_NAME)
    {
         client.println("ACK:" + getData());
      
    }else{
        client.println("\0");
    }

    client.flush();
    data = "\0";
  }
}
\end{lstlisting}
\cite{ESP32Ser71:online}

The Data is sent 10 times per second for evaluation purposes. 
However this will be reduced and aggregated. Since the position of the user is not as time sensitive, that a millisecond response time is necessary. This enables us to simplify the data transfer from the devices, which is a benefit only achievable by the current configuration of "Smart-sensors".

The Data is then saved onto a SD Card. This enables me to analyse the Data after measuring it and is simple to implement. This Concept works, however, it has some major flaws. 

\section{Flaws of Attempt A}

Since I wanted to build an all in one system, I tried to save all the data locally. This was e mental barrier, set by the first attempt, where everything had to be fit in a single belt. Where everything was collected and then sent from a single node to another device, with Bluetooth or a local WiFi hotspot. This "closed system" and single node idea hat to thrown out before achieving a real improvement.

Furthermore, during the development and testing of this concept further flaws arose. The client server network was, in its current implementation, not very stable. Additionally, the implementation required much more hardware than I would have liked. 
The Server needed a few modules for it to work, as shown below:


\begin{figure}[ht]
  \begin{center}
\includegraphics[width=\linewidth]{images/CommunicationDiagrammExplenation.png}
  \end{center}
  \caption{Client Server Network}
  \label{fig:ClientServer}
\end{figure}

This does not look like much but both of these elements, the RTC and the SD Card reader, add an additional point of failure, are quite complex and need to be configured. Additionally many different kinds of RTC and SD Card reader exists which do not have the exact same communication protocol or library. Which is a big flaw, when trying to create something open-source and useful. Since it restricts users and would require multiple different implementations. All this felt like quite the hassle and I soon realised I need to find a new way to communicate.

Beside these technical issues the data transfer during the development phase, the transfer and visualisation of the data was quite awkward. The device needed to be stopped and the data needed to be transferred manually from the SD-Card. This made the analysis very hard and static.

\begin{figure}[ht]
  \begin{center}
\includegraphics[width=0.5\textwidth]{images/CommunicationDiagrammLocal.png}
  \end{center}
  \caption{Communication Diagramm Local}
  \label{fig:CommunicationDiagrammLocal}
\end{figure}

After working with it for a while I tried to find a much simpler way. An implementation closer to the goal of an easy and affordable sensor module. To improve the client server implementation, I first thought of using a simple http request. Since The messages would have been sent on the Application layer and not the network layer (see Figure \ref{fig:Networkpacket}) the HTTP protocol would have had much more overhead. However, in theory, this also would mean simpler and more stable communication. This however did not feel right. Running a HTTP server on a microcontroller did not offer any apparent benefits over using a "real" server. Furthermore HTTP has way to much overhead for such a scenario. I realised that the solution, was something I already was quite familiar with. A simple MQTT Broker. 

\begin{figure}[ht]
  \begin{center}
\includegraphics[width=0.7\textwidth]{images/Screenshot_5.png}
  \end{center}
  \caption{Network Packet}
  \label{fig:Networkpacket}
\end{figure}
\cite{22Dissec90:online}


\section{Attempt B - Communication via MQTT}

After this first attempt I understood the received data and how to handle the message from the sensors.
The Communication in my new attempt was handled with a simple \acrshort{mqtt} Broker (\gls{MQTT}), which can be setup for free within seconds. For my endeavours I used cloudmqtt.com which is completely free and easy to apply. This MQTT broker could also run locally. However, this approach was not further investigated or tested. A local setup would be simple and but would currently not offer any real benefit.


\begin{figure}[ht]
  \begin{center}
\includegraphics[width=0.8\textwidth]{images/CommunicationDiagrammMQTT.png}
  \end{center}
  \caption{Communication Diagramm MQTT}
  \label{fig:CommunicationDiagrammMQTT}
\end{figure}

MQTT uses a simple publish subscribe protocol which I have already implemented a few times.
Below you see the whole setup and loop code, which is almost readable thanks to helper functions:
\begin{lstlisting}
void setup() {
  Serial.begin(115200);
  Wire.begin();
  USERID = getRegisterdUserid();
  setupMPU();
  setupWifi();
  client.setServer(mqtt_server, mqtt_port);
  client.setCallback(callback);

  strcpy(fullsubtopic, TOPIC); 
  strcat(fullsubtopic, USERID);
  strcat(fullsubtopic, SUB);
  
  strcpy(fullpubtopic, TOPIC); 
  strcat(fullpubtopic, USERID);
  //strcat(fullpubtopic, PUB);
  
  while (!client.connected()) {
   reconnect();
  }
  startMillis = millis(); 
  getData();
}

void loop () {
  client.loop();
  currentMillis = millis();  
  //get the number of milliseconds since the program started
  if (currentMillis - startMillis >= period)  
  //test whether the period has elapsed
  {
    Serial.println("trying to send");
    char * message = getMessage();
    publish(message);
    Serial.println(message);
    startMillis = currentMillis;  
    //IMPORTANT to save the start time of the current LED state.
  }
 getData();
 }
\end{lstlisting}

This loop collected the data into an average in "getData()" function
\begin{lstlisting}
void getData(){
  if(readByte(MPU6050_ADDRESS, INT_STATUS) & 0x01) { 
  // check if data ready interrupt
    readAccelData(accelCount);  // Read the x/y/z adc values
    getAres();
    ax = (float)accelCount[0]*aRes - accelBias[0]; 
    // get actual g value, this depends on scale being set
    ay = (float)accelCount[1]*aRes - accelBias[1];   
    az = (float)accelCount[2]*aRes - accelBias[2];  
    readGyroData(gyroCount);  // Read the x/y/z adc values
    getGres();
    gx = (float)gyroCount[0]*gRes - gyroBias[0];  
    // get actual gyro value, this depends on scale being set
    gy = (float)gyroCount[1]*gRes - gyroBias[1];  
    gz = (float)gyroCount[2]*gRes - gyroBias[2];  
    tempCount = readTempData();  // Read the x/y/z adc values
    temperature = ((float) tempCount) / 340. + 36.53; 
    // Temperature in degrees Centigrade
 
    accelVal[0] = ax;
    accelVal[1] = ay;
    accelVal[2] = az;
    gyroVal[0] = gx;
    gyroVal[1] = gy;
    gyroVal[2] = gz;
    setAvg();
    
   }
}
void setAvg(){
    for (int i = 0; i < 3; i++){
      avgAccelVal[i] = (avgAccelVal[i] + accelVal[i])/2;
      avgGyroVal[i] = (avgGyroVal[i] + gyroVal[i])/2;
  }
}
\end{lstlisting}
 and created a JSON from the collected data in the "getMessage()" function which was sent every second: 
\begin{lstlisting}
char* getMessage(){
  char* a = "{ \"id\": \"";
  char* b = "\", \"acc\":[";
  char* c= "], \"gyro\":[";
  char* d= "]}"; 
  char accelbuff[64];
  char gyrobuff[64];
  Serial.println("loading data to buffers");
  char* loc = accelbuff;
   size_t tempLen;
   int i = 0;
   for(i = 0; i < DIM(avgAccelVal)-1; ++i)
    {
        snprintf(loc, 12, "%f,", avgAccelVal[i]);
        tempLen = strlen(loc);
        loc += tempLen;
    }
    snprintf(loc, 12, "%f", avgAccelVal[i]);
   //snprintf(loc, 12, "%f", avgAccelVal[i+1]);
   tempLen = strlen(loc);
   loc += tempLen;
   loc = gyrobuff;
   for(i = 0; i < DIM(avgGyroVal)-1; ++i)
    {
        snprintf(loc, 12, "%f,", avgGyroVal[i]);
        tempLen = strlen(loc);
        loc += tempLen;
    }
  snprintf(loc, 12, "%f", avgGyroVal[i]);
  //snprintf(loc, 12, "%f", avgGyroVal[i+1]);
  tempLen = strlen(loc);
  loc += tempLen;
  strcpy(messagebuffer, a ); 
  strcat(messagebuffer, DEVICEID);
  strcat(messagebuffer, b);
  strcat(messagebuffer, accelbuff);
  strcat(messagebuffer, c);
  strcat(messagebuffer, gyrobuff);
  strcat(messagebuffer, d);
  return messagebuffer;
}
\end{lstlisting}
As you can see the creation of the message, was almost the most complex part of the implementation. This was due to the fact, that the "PubSubClient" Library \cite{knollear26:online}, used to communicate with the MQTT Broker, could not handle "arduino" strings. Therefore char pointers were needed, which are a quite complicated to work with. The high complexity also caused an error while concatenating the char pointers. The lines which are commented out, is where the error occurred. Data was transferred and the JSON message almost looked correct. However, since the variable "i" is incremented once more, when accessing the index of the array. The wrong address, avgAccelVal[3] instead of avgAccelVal[2] is accessed. Therefore, the last acceleremoter value, is actually the first gyroscope value. This occurred, since the two char arrays were allocated directly after each other. The last gyroscope value was pointed to a random address, therefore the value was random as well. 
This mistake was unfortunately only discovered when visualising the data, since C++ does not throw any errors when accessing items out of the array, and simply returns the values in the addresses accessed.
Nonetheless after some tinkering I finally managed to get a correct \acrshort{json} message (\gls{JSON}) over the MQTT Broker:

\begin{lstlisting}
{ 
    "id": "SENSOR-XSZ", 
    "acc":[-0.003835,0.001486,0.056012],
    "gyro":[0.056012,0.240598,0.038814]
}
\end{lstlisting}

\acrshort{protobuff} (\gls{Protobuf}) would be a great alternative, which I will try to implement in the future. Protobuf does take some time to set up but would simplify the data transfer greatly since bytes could be directly sent and would not need to be concatenated to a JSON. However this was not attempted in this thesis. 

Additionally I enabled all devices to be calibrated remotely. This since when I put these devices on I will need to calibrate them after they are set in position. This is quite a simple task with MQTT since the clients can simply subscribe to a topic, from which they get messages: 

\begin{lstlisting}
void callback(char* topic, byte* message, unsigned int length) {
  Serial.print("Message arrived on topic: ");
  Serial.print(topic);
  Serial.print(". Message: ");
  String command;
  
  for (int i = 0; i < length; i++) {
    Serial.print((char)message[i]);
    command += (char)message[i];
  }
  if(command == "calibrate"){
    calibrateMPU6050(gyroBias, accelBias); 
    // Calibrate gyro and accelerometers, load biases in bias registers  
    initMPU6050(); 
    Serial.println("MPU6050 initialized for active data mode....");
  }
}
\end{lstlisting}
This callback gets registered when connecting to the mqtt broker.

\subsection{Benefits Version 2}

The benefits of managing the messages via a MQTT Broker was, that I could get the messages and work with them from any device with an Internet connection. This means that any calculation heavy tasks can be done from a "real" computer which handles these much better. 

This also enabled me to visualise and analyse the data in real-time which made it much easier. 

Furthermore it also achieves the goal of being cheap and affordable. Since we now only need MPU6050 and microntrollers to send the data and this can be setup much easier than with an ESP32 "server".

This also means that there does not need to be any more logic on the devices itself since everything can be done from a Computer. The vibration device which is attached to a sensor pin, can also be activated via the MQTT Broker. 

The Fact that these sensors now can simply send data without knowing where they are sending their data to opens quite a lot of doors. I will get further into that in the chapter "Prospects".

\subsection{Backdraws Version 2}

The Obvious backdraw is, that the sensors need to have a constant Internet connection. This is obviously not great, especially when we try to develop something for an every usage since we are somewhat bound to a single place. 

Furthermore if used as a product, these sensors would need to be registered first before usage, this does increase complexity for a first usage however simplifies setup greatly.


\section{Identifying Posture}

The Biggest hurdle, was and still is identifying and correctly analysing the data. I have had several different aproaches on how to interpret and visualize the data. During my first attempt (Attemt A) i tried to visualized the data in two ways, with Python and Power BI.

Firstly I transformed the data into a CSV which is a commonly used format viable for many different applications

\subsection{Visualising with Python}

I tried to visualize the data using python since it was recommended on a few forums and seamed to be feasible for this task. It actually did work quite well and I have achieved within 2 Days with Python. I have never worked with python before and was happy that I managed to use and apply it within days. 

\includegraphics[width=\linewidth]{images/PyVisualisation.png}

Visualised you see my first attempt in showing movement of 3 sensors with the data of the accelometers. The sensors were attached to my shirt with scotch. This is not a perfect solution, which is visible in the different orientation of the green dots. 

Here I tried to add up the movement from the accelometers to see how this stacks up. The Data was not very helpful and i tried to anime the accelometer data directly to see how it changes, this was a bit clearer but still very hard to interpret since it was static data. 

To Visualize I used numpy and matplotlib which are quite handy but still took some time the get used to.
The data was simple display on a grapth (ax) from np arrays:

\begin{lstlisting}
ax.scatter3D(XSXAcc[:,0],XSXAcc[:,1],XSXAcc[:,2], c=XSXAcc[:,2], cmap='Reds')
ax.scatter3D(XSYAcc[:,0],XSYAcc[:,1],XSYAcc[:,2], c=XSYAcc[:,2], cmap='Blues')
ax.scatter3D(XSZAcc[:,0],XSZAcc[:,1],XSZAcc[:,2], c=XSZAcc[:,2], cmap='Greens')
\end{lstlisting}

\subsection{Visualising with Power BI}

\section{The Web App}

The static visualisation of data has a lot of issues, especially when the data is dynamic. To get a real understanding of the collected data, it would needed to be compared to a video. It was very hard comprehend what the effect, especiall of smaller movement had, on sensor. Therefore, I decided to implement a small web app which visualises the data in real time. 

This was achieved with a simple C\# \gls{MVC} website which subscribes to the MQTT broker. Here the switch to MQTT and the extra effort for this switch, was highly beneficial.

The website started small and was simply used to visualise the collected data:
\begin{figure}[h]
  \begin{center}
\includegraphics[width=\linewidth]{images/WebVisualisation_SIMPLE.png}
  \end{center}
  \caption{Simple Web Visualisation}
  \label{fig:SimpleWebVisualisation}
\end{figure}

Thanks to this visualisation I was quickly able to understand the data and add additional visualisations and make first conclusions which data was necessary for an in depth analysis. It became clear, that the data needed to be "transferred" into a much more readable format, like pitch, roll and yaw. The conversion was quite simple and needs only the acceleremoter data:
\vspace{-5pt}
\begin{lstlisting}
function getRoll(x, y, z) {
    pitch = 180 * Math.atan(x / Math.sqrt(y * y + z * z)) / Math.PI * 4;
    roll = 180 * Math.atan(y / Math.sqrt(x * x + z * z)) / Math.PI * 4;
    yaw = 180 * Math.atan(z / Math.sqrt(x * x + z * z)) / Math.PI * 4;
    return {
        "roll": roll,
        "pitch": pitch,
        "yaw": yaw
    }
}
\end{lstlisting}
\cite{Beginner65:online}


With this formula the position of the sensors was readable and elementary to visualise. This function is great for the implemented sensors and the current use case. It uses only the current measured values for calculations. Other formulas for positional analysis, like Quaternions \cite{MathsQua11}, are much more complex and use the sum of the measured data. This can cause drift, by small errors adding up. Which, especially when working with "amateur" sensors and long term measurements, can be a real problem.

When this formula is applied with the literal and live data which is sent from the sensor, still a lot of jitter is present. Due to this the data was not very readable or meaningful. To reduce this jitter, I decided decided to visualise the median of a 10 second measurement. Since the refresh rate is not very important for the current use case, this did not have any disadvantages. It became clear after some testing and discussion, that one sensor would not be enough, since its position does not offer sufficient information about the posture. (see Figure \ref{fig:BackPos})

\begin{figure}[h]
  \begin{center}
\includegraphics[width=\textwidth]{images/Backposition.png}
  \end{center}
  \caption{Back position}
  \label{fig:BackPos}
\end{figure}

Therefore, a minimum of two sensor would definitely be needed to make general conclusions from the data. This due to the fact, that posture can change on the lower or the upper back, as visualised in Figure \ref{fig:BackPos}). Which means it is possible that the upper back stays straight and the lower back is buckled or the other way around. Additional sensors would be optimal to get more detailed information about posture and the curvature of the spine. Also the shoulders are not tracked with the current sensor position. Therefore, it must be pointed out, that currently it is only possible to determine the positions of the lower and upper back. Not much further insight can be given with only two sensor. However, I consider this to satisfy the goals of this thesis. It offers enough insight to a casual wearer and has the potential to help improve posture. Therefore, also due to time restrictions and missing know-how, an in-depth analysis of the spine and curvature was not done in this thesis. To improve the overall understanding and visualisation of the data, three sensors were used during testing. The Idea was to calculate the degrees between each sensor to check whether the user has a bent back. This was later overturned since the curvature of the back does not give any insight in healthy posture \cite{SitUpSt77:online}. However, I might get back to that idea since it would enable the user to have more flexibility when moving. It might allow to increase the threshold, when the entire back is moved correctly.

Through the MQTT-Broker the data of three sensors is sent as JSON. In the JSON is the UID of the sensor with which it is identified. Then the data is parsed and passed through a javascript library (plot.ly) \cite{ModernAn18:online} to create the graphs. To simplify the analysis of the data I further also knew the position of each sensor:

\begin{figure}[ht]
  \begin{center}
\includegraphics[width=0.3\textwidth]{images/ChairVisualised.png}
  \end{center}
  \caption{Sensor position}
  \label{fig:SensorPos}
\end{figure}

After some research, testing and discussion with a physiotherapist, we realised, that the approach of trying to set a general goal for all users was not correct, in many ways. Different studies found that posture is a very individual thing and cannot be set globally \cite{SitUpSt77:online} therefore, we decided to implement functionality so that each user can set his/her own individual goals. I will get further into these decision in the chapter "Results".

\subsection{Functionality}

Thanks to distributed setup and the data transfer using a MQTT broker, the web app could not only be used to visualise the data. Additional functionality was added which uses the live data to help the user improve his/her posture. This was achieved with two, lets call them, modes. 

The first mode is the "Goal" mode, where the user can set a goal posture which he would like to achieve and keep. The user clicks to button "calibrate" and then the desired posture needs to be held for at least 10 seconds. 
Now every sensor has a calibrated optimal value. Each deviation of this optimal value is visualised, and when a threshold is reached a warning is triggered to the user. In the beginning of this project the warning was only visual. However a tactical feedback from the sensor itself was implemented later on.

These modes are visualised as below. In the circle in figure \ref{fig:TargetPosition} you see a red line and a green line. Here the avoid mode is visualised where the red line is the position the user would like to avoid and the green line is his current position.

\begin{figure}[ht]
  \begin{center}
\includegraphics[width=0.7\textwidth]{images/WebAppCircle.png}
  \end{center}
  \caption{Target position}
  \label{fig:TargetPosition}
\end{figure}

The second mode is very similar. However, a position the user would like to avoid is programmed and the sensor react when the user gets to close to this position rather than to far away. 

For the calculation of these modes each sensor was only analysed individually. However, this could be easily improved by calculating the overall changes of position, so to accumulate each sensors deviation in degrees. This would enable the web app to trigger a warning when the user is bent to much or too little.


\section{Hardware}

The Implementation with MQTT not only simplifies the communication but also reduces the hardware complexity greatly. The only parts needed are an MPU6050 a vibration unit and some sort the micro-controller. Since MPU and vibration unit are suitable with almost any micro-controller there are almost no restrictions left. Additionally even the accelerometer and gyroscope hardware could be changed. The only restriction to analyse the data, is that the data is sent in the same JSON format. 

The parts were not all soldered together. The first implementation with the belt, tought me, that soldering things together is not optimal for prototyping. So I decided to create modules which can be attached to each other using cables. (Figure \ref{fig:SensorsViaCable}) This meant the different parts can be reconfigured multiple times.


\begin{figure}[ht]
  \begin{center}
\includegraphics[width=0.8\textwidth]{images/PluginSensorCable.png}
  \end{center}
  \caption{Sensors Attached via cable}
  \label{fig:SensorsViaCable}
\end{figure}

Currently the hardware all together costs, even when bought in Switzerland and as a single unit, about 20 CHF. When bought from china, still as single units, the price can be reduced to about 12 CHF. I would estimate that the price could be lowered further, when buying in bulk, to about 5 CHF.


\section{Technical Difficulties}

Independent of the communication protocoll, there are some issues, or difficulties when trying to visualize and interpret accelometer and gyrometer data. 
