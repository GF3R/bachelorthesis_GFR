\chapter*{Current Attempt}
\label{chap:Technical CHallenges}
\renewcommand{\thesection}{\arabic{section}}
\setcounter{section}{0}

After my first attempts and some discussion with my supervisor i realized that the solution is to distribute the system and send the data wirelessly. This enables me to add additional sensors without having issues with the Wire library. However this increases the cost and complexity of each sensor slightly, since each sensor needs some sort of "brain", which handles the connection and communciation. 
However it reduces the complexity of the construction and the entire solution drasticly since it has single services talking ot each other and not an entier blob of logic. 

\section{Communication}

The communication will be firtly achieved via WIFI since I know the protocol and can get it running swiftly. Also the esp32 I have been using have WIFI natively enabled. 
An issue with wireless communication is the synchronisation of the data flow. To simplify the process and due to lack of knowhow and time to achieve a correct synchronisation I will ignore it, while knowing the data might be timeshifted. 
This might be an issue when visualising and analysing the data, but will be tackled then.

Forthermore for the communication via WIFI a simple "Client Server" protocoll is used which has allready been widely implement in ESP32.

Server Client Request: 
\begin{lstlisting}
String clientRequest(String input)
{
  Serial.println(input);
  String response = "\0";
  for (int i = 0; i < NUM_CLIENTS; i++)
  {
    WiFiClient client = server.available();
    client.setTimeout(50);
    if (client) {
      if (client.connected()) {
        client.println(input);
        data = client.readStringUntil('\r');  // received the server's answer
        Serial.println(data);
        if (data != "\0")
        {
          int Index = data.indexOf(':');
        
          CLIENT = data.substring(0, Index);
          ACTION = data.substring(Index + 1);
          Serial.println(data);
   
          if (CLIENT == "ACK")
          {
            response = ACTION;
          }

          //client.flush();
          //data = "\0";
        }
      }else{
        Serial.println("client not connected");
      }
    }
    else{
        Serial.println("client null or false");
      }
  }
\end{lstlisting}

Client Loop:

\begin{lstlisting}
void loop () {
  if (!client.connect(server, 80)) {
    while (WiFi.status() != WL_CONNECTED) {
      Serial.print(".");
      delay(500);
    }
    Serial.print("+");
    delay(100);
    return;
  }
  data = client.readStringUntil('\r');  // received the server's answer
  Serial.println(data);
  if (data != "\0")
  {
    int Index = data.indexOf(':');

    CLIENT = data.substring(0, Index);
    ACTION = data.substring(Index + 1);

    if (CLIENT == CLIENT_NAME)
    {
         client.println("ACK:" + getData());
      
    }else{
        client.println("\0");
    }

    client.flush();
    data = "\0";
  }
}
\end{lstlisting}

The Data will be sent 10 times per second for evaluation purposes. 
However I hope and think this can be reduced, since we hopefully will not need that much data to determine how to usere i positioned. This would enable us to aggregate the data on the devices, which is a benefit only achievable by the current configuration of "Smart-sensors".

\section{Sensors and Data}

\section{Identifying Posture}

The Biggest hurdle, was and still is identifying and correctly analysing the data. I have had several different aproaches on how to interpret and visualize the data. During my first attempt (Attemt A) i tried to visualized the data in two ways, with Python and Power BI.

Firstly I transformed the data into a CSV which is a commonly used format viable for many different applications

\subsection{Visualising with Python}

I tried to visualize the data using python since it was recommended on a few forums and seamed to be feasible for this task. It actually did work quite well and I have achieved within 2 Days with Python. I have never worked with python before and was happy that I managed to use and apply it within days. 

\includegraphics[width=\linewidth]{images/PyVisualisation.png}

Visualised you see my first attempt in showing movement of 3 sensors with the data of the accelometers. The sensors were attached to my shirt with scotch. This is not a perfect solution, which is visible in the different orientation of the green dots. 

Here I tried to add up the movement from the accelometers to see how this stacks up. The Data was not very helpful and i tried to anime the accelometer data directly to see how it changes, this was a bit clearer but still very hard to interpret since it was static data. 

To Visualize I used numpy and matplotlib which are quite handy but still took some time the get used to.
The data was simple display on a grapth (ax) from np arrays:

\begin{lstlisting}
ax.scatter3D(XSXAcc[:,0],XSXAcc[:,1],XSXAcc[:,2], c=XSXAcc[:,2], cmap='Reds')
ax.scatter3D(XSYAcc[:,0],XSYAcc[:,1],XSYAcc[:,2], c=XSYAcc[:,2], cmap='Blues')
ax.scatter3D(XSZAcc[:,0],XSZAcc[:,1],XSZAcc[:,2], c=XSZAcc[:,2], cmap='Greens')
\end{lstlisting}

\subsection{Visualising with Power BI}
