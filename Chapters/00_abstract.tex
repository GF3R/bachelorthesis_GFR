\chapter*{Abstract}
\label{chap:Abstract}
\renewcommand{\thesection}{\arabic{section}}
\setcounter{section}{0}
Back pain is a very common issue and can, in many cases, be traced back to a bad posture. Technical solutions to avoid bad posture already do exist. However, none of them seemed to be affordable, simple and open source. This is what has been inspiring my journey: trying to provide a simple and available solution, user friendly for a very common issue.\newline
To track and improve posture, I went with simple and affordable sensors and actors which are widely available. To analyse posture it was quickly obvious, that a gyroscope or/and an accelerometer are needed to determine how a user is positioned. During testing and exchanging with physiotherapists, it became clear, that with the current implemented algorithms and my limited know-how, one sensor alone was not enough for providing clear data. But, with a minimum of two sensors the curvature of the back can be detected with sufficient certainty. To transfer the data of these sensors I chose MQTT. It simplifies data communication and also reduces the specialized hardware necessary for this project, which further improves affordability and recreatability. With a simple web app the user can set personal targets for postures or postures to be avoided, which the web app visually, and the sensors tactically, enforce.\newline
The data gathered and the acquired experience of technical implementations are a great starting point for further development to portray individual healthy posture in the digital image. However, much more data and interdisciplinary work, with physiotherapists and other medical experts, will be needed to achieve the quality needed for such a venture.
