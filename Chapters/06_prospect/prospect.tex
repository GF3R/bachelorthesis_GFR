\chapter*{Prospects}
\label{chap:Porspects}
\setcounter{section}{0}

As mentioned much more has to be achieved, even so, a great base was made. However a lot of interdisciplinary work between computer scientists, medicinal experts and electric engineers would be needed to create a "product". This will never be a primary goal but it will be kept in mind.

\section{The Next Steps}

A lot more work and studies are necessary to prove, that this concept offers a real benefit to posture or even improve back pain. A lot of testing and data is needed to see whether these sensor can be used long-term and that the hardware is durable enough. These modules were merely tested for few hours.

The learning from this thesis will enable me to try and finalise the first version with a single sensor. Surely the precision would not be ideal. It might be enough for daily usage and has the potential to help with posture on a very elementary level. As discussed in the chapter "The Web App" a lot of information is missed with a single sensor.
Therefore, a lot of "field" testing must be done, to ensure a high enough certainty.

These potential of these sensors a much bigger use case than expected. With my current know-how and more time I might be able to measure more than posture. My partner is a physiotherapist and already had some great ideas where these sensors might be used. For example to measure bending of knees and other joints. 

Another ambitious goal is to visualise movement with the sensors. I am not sure if this is a possibility since the drift of the measurement could be drastic, however this is something I will definitely try to implement, since if possible, would open a lot of opportunities.

Adding an additional sensor as a reference point would offer a great improvement. Since it would enable users to wear and track posture even when working in a moving environment. The calculation  is probably quite complicated. Since the movement of the reference point has to be calculated in real time. This increases the complexity, since synchronisation and latency will need to be considered, which I was able to mostly ignore during this project.

Furthermore I will also try out other accelerometers and gyroscopes to determine precision and what would be optimal for my use case. This was not considered, since the MPU6050 does offer enough precision and comparing sensors requires a lot of know-how and time.