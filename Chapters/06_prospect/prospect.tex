\chapter*{Prospects}
\label{chap:Porspects}
\setcounter{section}{0}

As mentioned much more has to be achieved, even so, a great base was made. However a lot of interdisciplinary work between computer scientists, medicinal experts and electric engineers would be needed to create a "product". This will never be a primary goal but it will be kept in mind.

\section{The Next Steps}

The learning from this thesis might enable me to try and finalise the first version with a single sensor. Surely the precision would not be optimal however it might be enough for daily usage and could still help with posture.
This idea must be tested more in depth and would definitely would need a lot of "field" testing to ensure a high enough certainty before this could be counted as a viable use case. 

These sensors might have a much wider use case than expected and with my current know-how and more time I might be able to measure more than posture. My partner is a physiotherapist and already had some great ideas where these sensors might be used. For example to measure bending of knees and other joints. 

Another quite ambitious goal is to visualise movement with the sensors. I am unsure if this is a possibility since the drift of the measurement could be drastic, however this is something I will definitely try to implement, since if possible, would open a lot of opportunities.

A goal which is most likely achievable and I would consider a great improvement is adding an additional sensor as a reference point. This would enable users to wear and track posture even when working in a moving environment. This is probably quite hard to calculate out the movement especially due to synchronisation and latency which I was able to mostly ignore during this project.

Furthermore I will also try out other accelerometers and gyroscopes to determine precision and what would be optimal for my use case. This was not considered, since the MPU6050 does offer enough precision and comparing sensors requires a lot of know-how and time.