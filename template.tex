%
% Main document
% ===========================================================================
% This is part of the document "Project documentation template".
% Authors: brd3, kaa1
%

%---------------------------------------------------------------------------
\documentclass[
	a4paper,					% paper format
	10pt,							% fontsize
	twoside,					% double-sided
	openright,				% begin new chapter on right side
	notitlepage,			% use no standard title page
	parskip=half,			% set paragraph skip to half of a line
]{scrreprt}					% KOMA-script report
%---------------------------------------------------------------------------

\raggedbottom
\KOMAoptions{cleardoublepage=plain}			% Add header and footer on blank pages


% Load Standard Packages:
%---------------------------------------------------------------------------
\usepackage[standard-baselineskips]{cmbright}

\usepackage[ngerman,english]{babel}										% english hyphenation
%\usepackage[latin1]{inputenc}  							% Unix/Linux - load extended character set (ISO 8859-1)
\usepackage[ansinew]{inputenc}  							% Windows - load extended character set (ISO 8859-1)
\usepackage[T1]{fontenc}											% hyphenation of words with �,� and �
\usepackage{textcomp}													% additional symbols
\usepackage{ae}																% better resolution of Type1-Fonts 
\usepackage{fancyhdr}													% simple manipulation of header and footer 
\usepackage{etoolbox}													% color manipulation of header and footer
\usepackage{graphicx}                      		% integration of images
\usepackage{float}														% floating objects
\usepackage{caption}													% for captions of figures and tables
\usepackage{booktabs}													% package for nicer tables
\usepackage{tocvsec2}
\usepackage{lipsum}
% provides means of controlling the sectional numbering
%---------------------------------------------------------------------------

% Load Math Packages
%---------------------------------------------------------------------------
\usepackage{amsmath}                    	   	% various features to facilitate writing math formulas
\usepackage{amsthm}                       	 	% enhanced version of latex's newtheorem
\usepackage{amsfonts}                      		% set of miscellaneous TeX fonts that augment the standard CM
\usepackage{amssymb}													% mathematical special characters
\usepackage{exscale}													% mathematical size corresponds to textsize
%---------------------------------------------------------------------------

% Package to facilitate placement of boxes at absolute positions
%---------------------------------------------------------------------------
\usepackage[absolute]{textpos}
\setlength{\TPHorizModule}{1mm}
\setlength{\TPVertModule}{1mm}
%---------------------------------------------------------------------------					
			
% Definition of Colors
%---------------------------------------------------------------------------
\RequirePackage{color}                          % Color (not xcolor!)
\definecolor{linkblue}{rgb}{0,0,0.8}            % Standard
\definecolor{darkblue}{rgb}{0,0.08,0.45}        % Dark blue
\definecolor{bfhgrey}{rgb}{0.41,0.49,0.57}      % BFH grey
%\definecolor{linkcolor}{rgb}{0,0,0.8}     			% Blue for the web- and cd-version!
\definecolor{linkcolor}{rgb}{0,0,0}        			% Black for the print-version!
%---------------------------------------------------------------------------

% Hyperref Package (Create links in a pdf)
%---------------------------------------------------------------------------
\usepackage[
	pdftex,ngerman,bookmarks,plainpages=false,pdfpagelabels,
	backref = {false},										% No index backreference
	colorlinks = {true},                  % Color links in a PDF
	hypertexnames = {true},               % no failures "same page(i)"
	bookmarksopen = {true},               % opens the bar on the left side
	bookmarksopenlevel = {0},             % depth of opened bookmarks
	pdftitle = {Template fuer Bachelor Thesis},	   	% PDF-property
	pdfauthor = {brd3},        					  % PDF-property
	pdfsubject = {LaTeX Template},        % PDF-property
	linkcolor = {linkcolor},              % Color of Links
	citecolor = {linkcolor},              % Color of Cite-Links
	urlcolor = {linkcolor},               % Color of URLs
]{hyperref}
%---------------------------------------------------------------------------
% Set up page dimension
%---------------------------------------------------------------------------
\usepackage{geometry}
\geometry{
	a4paper,
	left=28mm,
	right=15mm,
	top=30mm,
	headheight=20mm,
	headsep=10mm,
	textheight=242mm,
	footskip=15mm
}
%---------------------------------------------------------------------------

% Makeindex Package
%---------------------------------------------------------------------------
\usepackage{makeidx}                         		% To produce index
\makeindex                                    	% Index-Initialisation
%---------------------------------------------------------------------------

% Glossary Package
%---------------------------------------------------------------------------
% the glossaries package uses makeindex
% if you use TeXnicCenter do the following steps:
%  - Goto "Ausgabeprofile definieren" (ctrl + F7)
%  - Select the profile "LaTeX => PDF"
%  - Add in register "Nachbearbeitung" a new "Postprozessoren" point named Glossar
%  - Select makeindex.exe in the field "Anwendung" ( ..\MiKTeX x.x\miktex\bin\makeindex.exe )
%  - Add this [ -s "%tm.ist" -t "%tm.glg" -o "%tm.gls" "%tm.glo" ] in the field "Argumente"
%
% for futher informations go to http://ewus.de/tipp-1029.html
%---------------------------------------------------------------------------
\usepackage[nonumberlist]{glossaries}
\makeglossaries
\newglossaryentry{Accelerometer}{name={Accelerometer},description={A sensor which detect acceleration}}

\newglossaryentry{Gyroscope}{name={Gyroscope},description={A sensor which detects orientation and angular velocity \cite{Gyroscop33:online}}}

\newglossaryentry{ESP32}{name={ESP32},description={Low cost, low power, microcontroller by espressif \cite{ESP32Ove27:online}}}

\newglossaryentry{microcontroller}{name={microcontroller},description={A small computer on a single chip}}


\newglossaryentry{SD}{name={Secure Digital},description={A memory card standard}}

\newglossaryentry{RTC}{name={RTC},description={Real Time Clock - Computer clock that keeps track of time \cite{Realtime16:online}}}


\newglossaryentry{MPU6050}{name={MPU6050},description={Six Axis gyroscope and acceleremoter motion tracking device \cite{MPU6050T29:online}}}


\newglossaryentry{IoT}{name={Internet of Things},description={Connected computing devices and digital machines}}

\newglossaryentry{lipo battery}{name={lithium polymer battery},description={Energy storage technology}}

\newglossaryentry{MQTT}{name={MQ Telemetry Transport},description={Machine to machine connectivity protocol \cite{MQTT46:online}}}

\newglossaryentry{JSON}{name={JavaScript Object Notation},description={"A lightweight data interchange format" \cite{JSON31:online}}}

\newglossaryentry{Protobuf}{name={Protocol Buffers},description={"Protocol buffers are Google's language-neutral, platform-neutral, extensible mechanism for serializing structured data" \cite{Protocol99:online}}}

\newglossaryentry{CSV}{name={Comma Separated Values},description={Text file that contains structured data}}


\newglossaryentry{MVC}{name={Model-View-Controller},description={Design pattern used to decouple user-interface (view), data (model), and application logic (controller)" \cite{ASPNETMV81:online}}}


\newglossaryentry{Power BI}{name={Power BI},description={A data visualsiation tool from Microsoft \cite{PowerBIM60:online}}}


\newglossaryentry{Python}{name={Python},description={High level programming language \cite{Welcomet27:online}}}

\newacronym{mqtt}{MQTT}{MQ Telemetry Transport}
 
\newacronym{iot}{IoT}{Internet of Things}

\newacronym{protobuff}{Protobuf}{Protocol Buffer}

\newacronym{json}{JSON}{JavaScript Object Notation}

\newacronym{csv}{CSV}{Comma Separated Values}

\newacronym{sd}{SD}{Secure digital}


\newacronym{rtc}{RTC}{Real Time Clock}
%---------------------------------------------------------------------------

% Intro:
%---------------------------------------------------------------------------
